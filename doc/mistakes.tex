\section{总结}
\subsection{代码复用}
在课题刚开始的时候,准备工作并不仅仅是学习,而应该是去看看已有的相关类
似的开源项目。在代码编写的时候,我们本应该尽可能利用已有的工具来完成一
些功能,虽然我们确实已经借助了一些开源工具来完成整个不属于我课题本身的
内容,比如使用了D3来完成动画,使用了angular以及typescript来组织整体的代
码。这些远远不够,只是在基础的框架上面打转,我们没有一开始就从其他的编
译器解析相关的项目入手,比如网络上已经存在的jison这样的开源项目,这是一
个很大的遗憾。如果能够更早意识到利用jison已经实现的算法,配合我们做动画
的思路,我们就可以更加专注于动画的设计上面,从而设计出更多更合理更吸引
人的动画。为什么这么说呢,算法本身就是很多人都已经实现了,课题的重心并
不是去重新实现这些算法,而是将这些算法可视化展现出来,然而最终我们却浪
费了大量的时间和精力去学习这些算法,重新实现这些算法。
\subsection{项目结构}
在程序设计的时候,因为想尝试更多的组织方式,所以中间有段时间去使用了更
新的一些开源类库,然而效果并不好。而在后来的实践中发现,在现有的基础上
去改变代码的结构,而不是通过替换底层我们依赖的框架,有时候的表现更好。
例如说,我们最后将代码使用 typescript 整理起来,依靠类与模块来把代码一
个一个独立出来,得到一个相对清晰的代码结构。当然,在界面展示那个层次上,
代码的可读性还不是那么好,但是我们相信,这些内容可以在后续不断改进中,
把他们处理得当。
\subsection{前人的研究}
我们在进行这项课题的研究的时候,忽略了其实导师教导过我们的事情,就是
着手开始一个课题的时候,要去先找找历史上,其他人对这个课题或者说是类似
课题的研究。这一步被我们省略了,或者是完全没有在一开始的时候意识到,这
是一个非常重要的疏漏。最后在整理论文的时候,其实我们已经看到了很多其他
科研人员研究的内容,他们对可视化的分析很多方面都比我们透彻。而如果一开
始,我们就已经吸收了他们的研究,那么就可以做到在巨人的肩上,看得更高更
远,也就能获得更好的成果。尽管如此,我们也不会就这样结束掉这个课题,而
会继续在这个课题上充当一个合格的维护者的角色,为后续参与这个课题的同学
们提供我们的思路以及我们在之前所遇到的各种问题,还有我们对待这个课题的
想法。我们相信,这一切都是非常有价值的,就如同他人的研究对我们自己的研
究也同样有价值一般。
\subsection{课题本身的局限性}
课题主要针对的是编译原理过程,主要是前端过程的可视化,而这一点,其实是
可以延伸到实际设计算法中使用的。课题实际实现的内容,可以用在实际词法、
语法分析器的生成,而不仅仅局限于,将其演示为动画就作为课题的结束。实际
中,可以将这一教学工具,更充分利用起来,可以实现诸如语法调试等等功能,
来帮助学生自己设计文法的时候进行可视化调试,更深刻理解编译原理前端的过
程,以及能够作为一些爱好者、文法研究人员的可视化工具。当然,这对于我们
开发者的要求非常高,需要我们本身对编译原理的理解够深刻,同时也是对交互
界面的理解足够深刻,才能完成。这一点内容,其实是作为一种遗憾而写在这里
的,我们希望,以后能有更多的人参与这方面的努力,从而让编译界不再令新人
恐慌,让更多的人,看到编译原理其实不是那么困难,甚至是十分有趣的一件事
情。
\subsection{内容的欠缺}
在这个课题中,其实最终的目的是为了让学生们学习编译原理的几个过程,那么
当当只是动画其实是不够的,也应该加入更多的内容才能真正让学生学起来。所
以,方便学生输入文法,进行更友好的文法错误提示(是的,不要惊讶,我们是
有文法错误提示,就是不够友好)。同时,如果能在系统中设计一些交互性的游
戏,配合着文法设计进行起来,配合着对算法的理解,那么,应该可以做出一个
非常有实质作用的系统。
