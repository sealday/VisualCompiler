\section{技术背景}
\subsection{框架的选择}
我们进行的这个课题,选择了几个方面的框架,一方面是开发工具集的选择,我
们使用了 webstorm 作为我们首选的开发工具,配合一些辅助的开发工具,包括
git、node等等。而前端展示部分,使用了bootstrap作为基础的展示框架,而
angular作为逻辑安排的框架,另一方面,还使用了,typescript语言,而不直
接使用javascript语言,这一点是为了更好和后续的团队进行配合。

框架的选择有一定的随意性,这点体现在它局限在我们团队的认知上,我们团队
由于只有有限的时间以及有限的软件经验,因此能够认识到的框架的数量以及深
度有限,因此在课题中选择的框架具有一定的随意性。然而,框架的选择也是经
过我们团队慎重考虑过的,在我们长期的web开发中选择了这个框架。

之所以选择了这些工具,分别都有其背后的原因,主要围绕当前的团队开发效率,
未来的维护工作,作为课题的可行性,时间以及空间概念上的考虑。具体内容过
多,与课题本身的关系不如其他方面紧密,暂时不展开讨论。
\subsection{现有的技术支撑}
\paragraph{简述} 
现有的比较常见的作为对编译器前端的支持的软件有很多,比较著名的就有lex、
flex、Yacc、Bison等等,虽然这些提到的都只是支持生成c语言的词法分析器、
语法分析器,但是现实中还有很多其他语言的实现,在web中也有支持生成
javascript的实现。其实在课题的最初,应该要充分调查现在web技术中的相关
实现,如果有可能的话,把我们的课题的目标系统直接建立在其它的技术之上,
而不用自己从头开始写过。
\paragraph{词法分析器} 
lex或者flex作为词法分析器生成器,是在这个课题中我们自己去实现一个能满
足可视化展示的词法分析器的一个有效的参考,可以通过他们的源代码以及相关
的文档,来探索词法分析器在业界的实现。除此之外,因为他们本身是词法分析
器生成器,代码的思路必然和词法分析器有所区别,而他们生成的特定的词法分
析器也因为是生成代码,造成可读性并不好的问题。因此,还可以参考一些业界
手写的词法解析器作为参考,比如流行的浏览器核心的webkit中,就有对
javascript的词法解析器的代码。实际上,我们本身的系统主要的目的是易于理
解,所以主要的代码参考还是来自于编译原理\cite{compiler}书本上的内容。
词法分析器还需要演示关于有穷自动机与无穷自动机的内容,这些内容的演示我
们主要参考了网络上的一些现有的实现。
\paragraph{语法分析器}
语法分析器是一个难点,虽然也是业界具有非常成熟的一系列算法,然而却也是
学生理解的困难点,另一方面,也有不少学生认为语法分析器能够被语法分析器
生成器所生成而不在意语法分析器的本身的原理,这在学习编译原理这门课中是
一种不利好的想法。实际上有Yacc和Bison作为主流的语法分析器生成器,而也
有大量大型的项目使用自己手写的语法分析器,这也证实了学习语法分析器的基
本原理是有利于理解现有的代码的。我们并没有完整实现语法分析器生成器,而
是,把语法分析器的几个步骤一一拆解出来,分别实现,并且辅助以不同的动画。

\subsection{测试的问题}

由于 UI 测试方面不熟悉,在课题中,并没有对 UI 的部分进行测试编写。而另
一方面,对于逻辑方面的测试,我们使用了web开发中常见的测试框架,jasmine。
这其实还需要讨论为什么选择这个框架,我们的选择原因是,它本身包含了比较
丰富的特性,而且社区的开发相当积极。虽然我们团队的成员都对测试并不是十
分熟悉,但是使用这个框架写起测试还是非常的方便有效的。
