\section{绪论}
\subsection{选题背景}
编译原理是在大学本科中非常重要的一门课,然而,它的难度也相对较大,虽然
说论起算法来说,并不如算法课那样高难,但是它具有工程意义上的复杂性,并
不能都用简介的算法来表述清楚,其工作原理很难被学生直观理解和掌握。而这
个选题的目的就是,制作一个可视化的编译原理课件,采用非传统展示类的工具
来制作这个可视化,而是使用崭新的 web 技术,通过背后强大的语言和各大类
库的支撑,能够实现一个既能展示丰富内容的编译过程可视化,又能够激发同学
主动参与学习过程的软件。
\subsection{选题目标}
\subsubsection{主要目标}
实现一个可以供学习编译原理课程的师生使用的基于web的可视化编译原理过程
的系统。其主体是一个web程序,可以运行在现代的浏览器上,并且也能通过现
代常用的包装方式,成为一个桌面上的本地程序甚至是手机上的APP,并且提供
一个可选的后台,用于保存学生的文法、老师的参考文法以及其他更多的内容。
选题的基本目标是,能够在这个系统上面进行一个简单的左递归消除、提取左公
因式、生成预测分析表、生成LL分析器,选题的最终目标是,实现编译原理课程
中所涉及的所有算法的可视化。而这之外的功能,属于选题的附属功能,是教学
实践中的最佳的补充,使得整个系统能更好融入日常教学系统,而不是作为一个
图形化工具的辅助存在。另一方面,选题的目标系统还应该具有可插拔的特点,
并不是指系统本身可以插拔新的内容,而是指的系统本身的各种实现,可以应用
于现有的教学系统中,这也就是说,系统的每一部分可以剥离出来,而成为其他
系统的组件,这样可以拓展选题的目标系统的可用性,减少重复的劳作。
\subsubsection{附属目标}
由于选题本身是一个应用型的课题,那么,文档就是一个很重要的部分,论文本
身并不能作为一个有效的文档,它只是论述了选题本身的内容。程序的文档我们
粗略分成了开发者文档与使用者文档。尽管我们在设计系统的时候,以我们的直
觉来使得这个系统使用起来符合学习计算机的师生的直觉,然而我们自己的使用
习惯与其他人的使用习惯并不能一致,因此我们需要产出一个使用说明书,用于
系统的使用说明。另一方面,我们还要使得这个系统得以长期维护下去,而不是
当我们团队在这个选题结束的时候,这个系统就已经失去了维护,那么它的价值
就变得非常有限。所以,我们还会准备一份开发者文档,用于叙述选题目标系统
的各个有价值的环节,论述选题目标系统的架构,使用的类库以及主要的设计思
想等等,这一部分内容在论文也有所体现,因此,论文本身也是一份供给开发
者参考的文献。
\subsection{技术背景}
\subsubsection{基础框架与开发工具}
我们进行的这个课题,选择了几个方面的框架,一方面是开发工具集的选择,我
们使用了 webstorm 作为我们首选的开发工具,配合一些辅助的开发工具,包括
git、node等等。而前端展示部分,使用了bootstrap作为基础的展示框架,而
angular作为逻辑安排的框架,另一方面,还使用了,typescript语言,而不直
接使用javascript语言,这一点是为了更好和后续的团队进行配合。

框架的选择有一定的随意性,这点体现在它局限在我们团队的认知上,我们团队
由于只有有限的时间以及有限的软件经验,因此能够认识到的框架的数量以及深
度有限,因此在课题中选择的框架具有一定的随意性。然而,框架的选择也是经
过我们团队慎重考虑过的,在我们长期的web开发中选择了这个框架。

之所以选择了这些工具,分别都有其背后的原因,主要围绕当前的团队开发效率,
未来的维护工作,作为课题的可行性,时间以及空间概念上的考虑。具体内容过
多,与课题本身的关系不如其他方面紧密,暂时不展开讨论。
\subsubsection{课题相关技术}
\paragraph{简述} 
现有的比较常见的作为对编译器前端的支持的软件有很多,比较著名的就有lex、
flex、Yacc、Bison等等,虽然这些提到的都只是支持生成c语言的词法分析器、
语法分析器,但是现实中还有很多其他语言的实现,在web中也有支持生成
javascript的实现。其实在课题的最初,应该要充分调查现在web技术中的相关
实现,如果有可能的话,把我们的课题的目标系统直接建立在其它的技术之上,
而不用自己从头开始写过。
\paragraph{词法分析器} 
lex或者flex作为词法分析器生成器,是在这个课题中我们自己去实现一个能满
足可视化展示的词法分析器的一个有效的参考,可以通过他们的源代码以及相关
的文档,来探索词法分析器在业界的实现。除此之外,因为他们本身是词法分析
器生成器,代码的思路必然和词法分析器有所区别,而他们生成的特定的词法分
析器也因为是生成代码,造成可读性并不好的问题。因此,还可以参考一些业界
手写的词法解析器作为参考,比如流行的浏览器核心的webkit中,就有对
javascript的词法解析器的代码。实际上,我们本身的系统主要的目的是易于理
解,所以主要的代码参考还是来自于编译原理\cite{compiler}书本上的内容。
词法分析器还需要演示关于有穷自动机与无穷自动机的内容,这些内容的演示我
们主要参考了网络上的一些现有的实现。
\paragraph{语法分析器}
语法分析器是一个难点,虽然也是业界具有非常成熟的一系列算法,然而却也是
学生理解的困难点,另一方面,也有不少学生认为语法分析器能够被语法分析器
生成器所生成而不在意语法分析器的本身的原理,这在学习编译原理这门课中是
一种不利好的想法。实际上有Yacc和Bison作为主流的语法分析器生成器,而也
有大量大型的项目使用自己手写的语法分析器,这也证实了学习语法分析器的基
本原理是有利于理解现有的代码的。我们并没有完整实现语法分析器生成器,而
是,把语法分析器的几个步骤一一拆解出来,分别实现,并且辅助以不同的动画。
\subsubsection{测试技术}
在web的UI测试,其实在业界已经有比较成熟的方案,但是我们并没有采用自动化
UI测试的方案,而是选择了手工的UI测试。因为课题本身的系统还不方便进行自
动化的UI测试,而且系统界面本身比较没有很多的内容,手工测试即可满足要求。

对于逻辑方面的测试,我们使用了web开发中常见的测试框架,jasmine。它自身
包含了大量我们所需要的测试工具,包括expect、mock等等,其开发者团队开发
相当积极,在StackOverflow社区上面的问答数量也相当多,因此我们最终选择
使用这个测试框架进行测试。我们团队的成员也都是刚刚接触这个框架,但是已
经能很熟练使用这个框架进行测试。