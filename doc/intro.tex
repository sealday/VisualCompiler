\section{引言}
\subsection{选题背景}
编译原理是在大学本科中非常重要的一门课,然而,它的难度也相对较大,虽然
说论起算法来说,并不如算法课那样高难,但是它具有工程意义上的复杂性,并
不能都用简介的算法来表述清楚,其工作原理很难被学生直观理解和掌握。而这
个选题的目的就是,制作一个可视化的编译原理课件,采用非传统展示类的工具
来制作这个可视化,而是使用崭新的 web 技术,通过背后强大的语言和各大类
库的支撑,能够实现一个既能展示丰富内容的编译过程可视化,又能够激发同学
主动参与学习过程的软件。
\subsection{选题目标}
\paragraph{主要目标}
实现一个可以供学习编译原理课程的师生使用的基于web的可视化编译原理过程
的系统。其主体是一个web程序,可以运行在现代的浏览器上,并且也能通过现
代常用的包装方式,成为一个桌面上的本地程序甚至是手机上的APP,并且提供
一个可选的后台,用于保存学生的文法、老师的参考文法以及其他更多的内容。
选题的基本目标是,能够在这个系统上面进行一个简单的左递归消除、提取左公
因式、生成预测分析表、生成LL分析器,选题的最终目标是,实现编译原理课程
中所涉及的所有算法的可视化。而这之外的功能,属于选题的附属功能,是教学
实践中的最佳的补充,使得整个系统能更好融入日常教学系统,而不是作为一个
图形化工具的辅助存在。另一方面,选题的目标系统还应该具有可插拔的特点,
并不是指系统本身可以插拔新的内容,而是指的系统本身的各种实现,可以应用
于现有的教学系统中,这也就是说,系统的每一部分可以剥离出来,而成为其他
系统的组件,这样可以拓展选题的目标系统的可用性,减少重复的劳作。
\paragraph{附属目标}
由于选题本身是一个应用型的课题,那么,文档就是一个很重要的部分,论文本
身并不能作为一个有效的文档,它只是论述了选题本身的内容。程序的文档我们
粗略分成了开发者文档与使用者文档。尽管我们在设计系统的时候,以我们的直
觉来使得这个系统使用起来符合学习计算机的师生的直觉,然而我们自己的使用
习惯与其他人的使用习惯并不能一致,因此我们需要产出一个使用说明书,用于
系统的使用说明。另一方面,我们还要使得这个系统得以长期维护下去,而不是
当我们团队在这个选题结束的时候,这个系统就已经失去了维护,那么它的价值
就变得非常有限。所以,我们还会准备一份开发者文档,用于叙述选题目标系统
的各个有价值的环节,论述选题目标系统的架构,使用的类库以及主要的设计思
想等等,这一部分内容在论文中也有所体现,因此,论文本身也是一份供给开发
者参考的文献。