\section{需求和功能点}
这个部分详细论述了需求和功能点,为了给读者提供一个更全面的课题介绍。
\subsection{词法分析的动画设计}
\begin{itemize}
\item 词法分析动画设计,能够将词法单元、模式以及词素从动画上面说明清楚,
  能够实现适当的提示,包括鼠标移动上去的提示,本身的字幕提示以及任何其
  他合理的提示。
\item 输入缓冲动画设计,说明清楚缓冲的作用,在动画中要能够清晰标明指针
  的位置,缓冲区的特点,能够让用户自定义输入文本的内容,也能够提供从不
  同的地方输入文本的内容的功能,详细的要求参见对文法输入的要求。可以提
  供一定的算法步骤,进行算法演示,这部分可选,因为输入缓冲的目的是让同
  学理解输入缓冲,而输入缓冲的算法实践本身比较简单。
\item 词法单元的归约的动画设计,使用动画来说明词法单元归约的各种名词,
  即通过一个动画,展示串的各个部分分别指的是哪些内容,通过动画,来说明
  语言上的运算是怎么样子的。这里可以适当加上对正则匹配过程的动画演示,
  可以参考已有的开源项目,不一定做成动画,可以做成匹配过程可以看到的匹
  配元素,就已经能够比较清晰说明正则匹配的内容。
\item 词法单元的识别的动画设计,用一个动画,配合词素、词法单元名字、属
  性值的内容,来演示词法单元配合的时候应该如何去识别每一个词法单元,他
  们是怎么被标记出来的可以做出成动画,辅助一些流水线的效果。另外还需介
  绍基于状态转换图的词法分析的体系结构,这部分可以用动态的状态图来展示
  出来它的工作效果,可以辅以代码来查看当前执行的情况。
\item 有穷自动机的动画设计,这部分内容基础要求是通过算法步骤一步步展示
  如何进行子集构造,在这个基础上,应该能完成更高的目标,利用算法步骤得
  到的结果,来动态构造一个动态图。最后应该要点名字符串的算法效率,可以
  做一个在线的测试给出一个测试结果,让同学们能直观的理解字符串处理的效
  率。
\end{itemize}
\subsection{语法分析的动画设计}
\begin{itemize}
\item 语法分析树的动画设计,这部分不具体讲解算法,因此就展示任意给定的
  一段代码,再给定一个文法,展示下一个语法分析树的展开,预先应该存有多
  个代码段以及匹配的文法。而且,能够在一定范围内处理异常的情况,指的不
  是能修正错误,而是如果发生文法和代码不匹配的情况,能够告之用户这俩者
  之间有不匹配的情况,可以根据后面的算法,指明在匹配到哪里的时候出现错
  误。
\item 左递归消除的动画设计,这部分动画应该直接包括直接左递归和间接左递
  归的情况。能够用动画展示出来算法的每一步骤针对的是哪些元素,并且可以
  指示算法当前进行到的位置,以及可以停止获取当前的各项变量的情况。
\item 提取左公因子的动画设计,这个算法内容比较简单,动画可以按照算法的
  进程,演示每一个公因子被找到的过程以及被提取的过程。
\item FIRST和FOLLOW集合构造的动画设计,根据构造FIRST集和FOLLOW集的算法,
  依次计算每一个非终结符号的集合,可以以多个集合图的方式来演示动画,表
  现出如何找到这些集合的过程。
\item LL(1)文法的动画设计,对于这个文法,我们可以设计出一个简短的动
  画来说明它的内涵。另一个要求是做出它的预测分析表,用FIRST和FOLLOW集
  合来逐步构建一个,在动画中展示出,如何构建每一行每一列的元素,动画应
  与输入符号相关联的算法执行情况。可以设计出栈的动画,方便用户可以查看
  到进出栈的情况。
\item LR(0)语法分析的动画设计,展示自底向上的分析过程。
\item LR(1)语法分析的动画设计
\item LALR语法分析的动画设计
\end{itemize}
\subsection{文法管理}
\begin{itemize}
\item 支持从多种渠道输入文法,分别应该有直接输入文法,从文件中获取文法,
  从URL中获取文法,其中文件中获取文法应该支持直接拖拽来上传文法以及从
  文件管理器中选择文法这俩种方式。
\item 支持文法的保存与记录,可以对文法进行命名和归类,可以分享文法给其
  他人,可以对文法进行评注以及说明。
\item 支持对文法进行报错提示,指的是文法不符合既定的格式,具体的格式要
  求在程序内说明。
\item 
\end{itemize}