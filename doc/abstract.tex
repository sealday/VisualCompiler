\centerline{\textbf{\huge{\hei \小二 编译原理可视化课件的动画设计}}}
% 临时设置,应该包装起来会更合适
\vskip 2.5ex
\phantomsection
\begin{cnabstract}
	
编译原理是大学本科学习过程中的一门重要但难以掌握的一门学科。编译器
我们一般可以将其分成前端后端俩个部分,本科学习中主要学习了编译器中的前端部分。前端的内容在业界已经有很多成型的算法,不过学生在理解这些算法的时候,普遍存在着难以理解、容易混淆、形不成概念的问题,缺乏直观的认知,同时也会在细节的地方掌握不好。
		
这个课题的提出就是为了实现一个基于Web的可视化编译原理的课件,通过生动有趣的动画演示,实时的环境变量信息,清晰明了的算法步骤指示,将原本枯燥难以理解的课本内容,通过这些传达学生。这样有效解决学生学习枯燥无味的问题,增进学生学习编译原理的热情,也能够让老师更加容易指导学生学习。借助于web的力量,让任何人都可以在任何系统任何时刻都可以访问到这个系统,让学习更加方便无障碍。

\textbf{关键字:}编译原理、可视化、web系统、动画设计
\end{cnabstract}
\addcontentsline{toc}{section}{摘要}
\newpage

\phantomsection
\begin{enabstract}
Compiler theory is the college is very important and difficult to
master a course, it is divided into two parts front and back of
undergraduate study is mainly learning compiler front-end, but the
contents of the front end, although the industry has there are many
forming algorithm, but the students understand these algorithms time,
the prevalence of the problem is difficult to really learn, and in
most cases only a rough impression, but there is no intuitive
understanding and grasp of the details. The issue raised is to
implement a system that can effectively solve this problem visually,
but also try to stimulate students' enthusiasm for learning compiler
theory, so learning compiler theory is no longer boring .

\textbf{Keywords:} Compiler, Visualization, web system, animation design
\end{enabstract}
\addcontentsline{toc}{section}{Abstract}
\newpage
